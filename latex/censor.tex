\documentclass{article}

\usepackage{sectsty}
\usepackage{amsmath}
\usepackage{amsfonts}
\usepackage{nameref}

\raggedright
\sectionfont{\fontsize{12}{10}\selectfont}
\subsectionfont{\fontsize{10}{8}\selectfont}

\begin{document}

\section{Intro}

Our current method makes two simplifying assumptions:

\begin{enumerate}
    \item \textbf{Perfect emission}: nodes that have not yet been emitted are
        emitted with probability 1 on first hit

    \item \textbf{Perfect censoring}: nodes that have been previously emitted
        are censored with probability 1 on all future hits
\end{enumerate}

The following method generalizes to allow imperfect censoring or emissions with
a fixed probability.

\section{Method}

Let $p_{emit}$ be the probability of emitting a node when it is encountered on a
random walk and has not been emitted before. Let $p_{censor}$ be the probability
of censoring a node when it is encountered on a random walk and has been
emitted before. Both parameters should typically be close to 1.

\vspace{5mm}

Previously, to calculate $P(X_{t+1}|X_1...X_t)$ we treat the set of previously
emitted nodes, $A=\{X_1...X_t\}$, as non-absorbing nodes and the set of
previously un-emitted nodes, $B=\{X_{t+1}...X_k\}$, as absorbing nodes ($k$ is
the number of elements in $X$).

\vspace{5mm}

Let $P$ be the transition matrix of graph $G$. We re-arrange the states of $P$ into:

\[
    P'=
    \begin{bmatrix}
        Q & R \\
        S & T
    \end{bmatrix}
\]

where $Q$ denotes the transitions from $A$ to $A$, $R$ denotes transitions from
$A$ to $B$, $S$ denotes transitions from $B$ to $A$, and $T$ denotes
transitions from $B$ to $B$. To account for imperfect censoring, we can treat
$A$ as non-absorbing with probability $p_{censor}$ and as absorbing with
probability $1-p_{censor}$. Similarly, to account for imperfect emission, we can
treat $B$ as absorbing with probability $p_{emit}$ and as non-absorbing with
probability $1-p_{emit}$. We compose new matrices:

\[
    Q'=
    \begin{bmatrix}
        p_{censor} \cdot Q & (1-p_{emit}) \cdot R \\
        p_{censor} \cdot S & (1-p_{emit}) \cdot T
    \end{bmatrix}
\]

\[
    R'=
    \begin{bmatrix}
        (1-p_{censor}) \cdot Q & p_{emit} \cdot R \\
        (1-p_{censor}) \cdot S & p_{emit} \cdot T
    \end{bmatrix}
\]

\[
    P''=
    \begin{bmatrix}
        Q' & R' \\
        0  & I
    \end{bmatrix}
\]

\vspace{5mm}

$Q'$ denotes transitions to all non-absorbing nodes that will not be emitted on
the random walk. $R'$ denotes transitions to all absorbing nodes that will be
emitted on the random walk.  Note that if $P$ is $n$ x $n$ large, $P''$ is $2n$
x $2n$ large to allow each node to be both absorbing and non-absorbing.

\vspace{5mm}

Given values for $p_{emit}$ and $p_{censor}$, we can now solve as we have been
doing. Or we might want to define a prior distribution for each parameter to
marginalize over.

\section{Discussion}

From a psychological perspective, imperfect censoring is necessary to account
for perseverations. $p_{censor}$ reflects some monitoring component, where
lower values correspond to poorer monitoring. It could also be used to model
disruptions to working memory. Imperfect emission may reflect a weak connection
between semantic and lexical (or motor) nodes. It could be used to model
certain types of anomia. Ideally, we would want to perform inference to recover
these parameters, since they have some psychological meaning.


\vspace{5mm}

From an ML perspective, perfect censoring may throw out informative data when
constructing a graph. Otherwise, if the raw data contain perseverations, we
must throw them out as if they didn't happen, distorting the data. On the other
hand, imperfect emission does not throw out any data, but may provide better
fits if the underlying process does contain imperfect emissions.

\end{document}

\documentclass{article}

\usepackage{sectsty}
\usepackage{amsmath}
\usepackage{amsfonts}
\usepackage{nameref}

\raggedright
\sectionfont{\fontsize{12}{10}\selectfont}
\subsectionfont{\fontsize{10}{8}\selectfont}

\begin{document}

\section{Generating an candidate graph}
We want to generate a set of candidate graphs ($G$s) and see how likely each is
to result in an observed $X$. The process for creating a potential graph
involves creating a possible random walk $Z$ of length $m$ that results
in $X$ of length $n$, and simply drawing edges between each adjacent
pair of nodes in $Z$; i.e., an edge between $Z_{i-1}$ and $Z_i$ for all
$i$ from $2...m$. $Z_1$ and $Z_2$ are constrained to $X_1$ and $X_2$,
respectively. For $i=3...m$, we set

\begin{equation}
    Z_i = 
    \begin{cases}
        \text{random element} \in \{X_1...X_{k-1}\} & \text{with probability } \theta \\
        X_{k}                 & \text{with probability } 1-\theta
    \end{cases}
\end{equation}

where $k$ is the next unobserved element in $X$. This process is repeated
until all elements in $X$ have been observed, i.e. when $k=n$.

\vspace{5mm}

To construct a candidate graph from multiple $X$, simply construct a
$Z$ for each $X$ using the same procedure.

\section{Computing graph probability}

Here is the original formulation for calculating graph probability from a single
$X$:

\begin{equation}
    \label{eq:likelihood_onex}
    P(G|X)=\sum_{Z}{P(G,Z|X)} \propto \sum_{Z}P(X|G,Z)P(Z|G)P(G)
\end{equation}

Since $X$ is deterministic from $Z$, the first term
$P(X|G,Z)$ is either zero or one, i.e.:

\begin{equation}
    P(X|G,Z)=\text{\Large{1}}_{\mathbb{Z}}{Z}
\end{equation}

where $\mathbb{Z}$ is the set of all possible $Z$s on $G$ that reduce to $X$
and \large{1} \normalsize is the indicator function. We enforce this by only
only sampling over $Z$s that produce $X$. Currently, we are assuming all graphs
are equally likely, and so $P(G)$ falls out. If you recall, I tried
implementing a small-world prior that didn't help, but frankly I was not
confident in my ability to implement this properly (i.e., how to calculate the
prior probability of a particular graph with small-world coefficient
$S^{\scriptscriptstyle{WS}}$ (Humphries, 2008).

\vspace{5mm}

So the meat of the computation is computing $P(Z|G)$ as:

\begin{equation}
    \label{eq:underflow}
    P(Z|G)=P(Z_1|G)\prod_{i=2}^{m} P(Z_{i}|Z_{1:i-1},G)
\end{equation}

I believe at the moment, $P(Z_1)$ is uniform and so since all graphs have the
same number of nodes, this falls out. It should be the stationary probability
of $Z_1$, but it still wouldn't matter for our purposes since the starting
point is fixed by $X_1$. To compute the likelihood of a graph with multiple
$X$, we extend Equation \ref{eq:likelihood_onex} further:

\begin{equation}
    P(G|X^1,X^2) \propto \sum_{Z^1 \in \mathbb{Z}^1} P(Z^1|G) \sum_{Z^2 \in \mathbb{Z}^2} P(Z^2|G)
\end{equation}

and so forth for more $X$.

\vspace{5mm}

Since computing $P(Z|G)$ in Equation \ref{eq:underflow} typically results in
underflow, we can approximate it using the log-sum-exp trick:

\begin{equation}
%    P(Z|G) \approx A + log \sum e^{-A} 
\end{equation}

\section{Generating a sample of $Z$s that result in $X$}

Since the likelihood term contains an infinite sum over $Z$, we need to
generate a representative sample of $Z$s to estimate this quantity. To generate
a $Z$ on graph $G$ that results in $X$, we take a random walk on $G$ starting
from $X_1$, while constraining the walk such no path to $X_j$ is taken before
hitting $X_i$ for all $j>i$. The walk is complete when we have a $Z$ that
results in $X$.

\section{Incorporating inter-item response times}

One problem with the approach above is that the procedure prefers sparse
graphs. That is, it seems to prefer the $RW$ solution used by Jun (submitted),
which treats the censored list $X$ as an uncensored list $Z$, and constructs a
graph with edges between each pair of adjacent nodes. The intuitive reason for
this is that the most sparse graph constructed from one $X$ will result in
shorter $Z$ sequences on average, since the probability of directly
transitioning between $Z_i$ and $Z_{i+1}$ is $.5$ for all recurrent nodes, the
maximum possible for an undirected graph.

\vspace{5mm}

This problem is alleviated somewhat by using multiple $X$, but it does not
appear to solve the problem entirely. By taking into account inter-item
response times (IRTs), we can penalize short $Z$s for not conforming to the
true pattern of IRTs. For example, in the most sparse graph constructed from
one $X$, it is predicted that all IRTs have the same approximately normal
distribution, but human data does not reflect this.

\vspace{5mm}

Let $I=\{I_1...I_{n-2}\}$ be a vector of observed inter-item response times.
Let $H=\{H_1...H_{n-2}\}$ be a corresponding vector where each item denotes the
number of hidden nodes between any two adjacent items (first hits) in $X$ for a
given $Z$. These vectors can instead be of length $n-1$, however $H_1$ will
always be $0$ because there cannot be any hidden nodes between $X_1$ and
$X_2$---therefore, its informative value is suspect. The vector $I$ should
correlate highly with the expected value of each item in $H$, i.e.
$E(H)=\{E(H_1)...E(H_{n-2})\}$.  

I can think of at least two ways to approach this, though I am sure there are
others and/or that these methods can be refined. I'll discuss both methods and
then go over the pros and cons.

\subsection{Method 1}
\label{sec:method1}

Previously, we generated a sample of $Z$s on graph $G$ that result in an
observed $X$. From this sample, we also form a distribution of hidden nodes
between any two adjacent nodes in $X$. That is, for any graph $G$ we know not
only $E(H)$ but also the full distribution of any $H_i$. Thus we can compute,
for any two items $I_i < I_j$, the probability that $G$ will generate a walk
$Z$ (that results in $X$) where $H_i < H_j$.

\vspace{5mm}

My formalism is not quite right, so I can't quite frame the full equation.
However the gist is to estimate $P(I|G,X)$ as $P(ord(H)=ord(I)|G,X)$ where
$ord(x)$ is the ordinal representation of $x$. This is done by:

\begin{equation}
    P(I|G,X) = \prod_{i,j}^{n-2} P(H_i<H_j|G,X) \text{ for all $I_i < I_j$} 
\end{equation}

The notation needs some work but I think this makes sense conceptually.

%\begin{equation}
%    P(G|I,X) = \sum_{Z} P(G,Z|I,X) = \sum_{Z} P(X|Z,G,I) P(I|Z,G) P(Z|G) P(G) 
%\end{equation}

\subsection{Method 2}
\label{sec:method2}

The second method is to look directly at the correlation between $I$ and the
$H$ corresponding to a specific $Z$. The $Z$ is more likely when
the correlation indicates a better fit, i.e. when the standardized sum of the
square residuals (SSR) is small. This is a little easier to formalize:

\begin{equation}
    P(G|I,X) = \sum_{Z} P(I|Z,G,X) P(X|Z,G) P(Z|G) P(G)     
\end{equation}

Here, the new term that we are estimating using the SSR is $P(I|Z,G,X)$.
Essentially, this term offsets the $P(Z|G)$ term so that we now don't
necessarily prefer short $Z$ sequences if the IRTs don't match $I$. How do we
know what a good SSR is? That's tricky---one solution is to compare it to the
SSRs generated from toy graphs, e.g. a subset of the animal semantic network
with the same number of nodes. We can compute this beforehand and form a
distribution of SSRs that are expected from reconstructing $X$.

\subsection{Benefits and drawback of each method}

There are several benefits to Method 1. One major benefit is that is does not
require any sort of prior expectations about what the graph should look like.
That is, it does not make use of the previously constructed animal semantic
network. Another major benefit is that it can actually be solved analytically,
rather that using random samples. Jun (submitted) shows in Theorem 3 that,
given a graph $G$, it is possible to solve analytically for the probability of
transitioning between any $X_i$ and $X_{i+1}$ (Doyle \& Snell, 1984). This
allows computing $P(X|G)$ without generating a sample of $Z$s, which is by far
the largest computational bottleneck in our procedure. A similar procedure can
generate the expected length of a random walk before being absorbed. Thus, we
can compute both $P(X|G)$ and $P(I|G)$ without doing expensive Monte Carlo
simulations.

\vspace{5mm}

There are two primary drawbacks to using Method 1. First is that it considers
only an ordinal representation of the IRTs. Whether the difference between two
IRTs is 4s and 4.01s or 4s and 100s, the model will treat them the same. The
other drawback is that it treats the IRTs as independent from the $Z$s that
created them. That is, the model finds the likelihood of generating $X$ on $G$
and the likelihood of generating $I$ on $G$, but the two are independent. I
don't know how important this is.

\vspace{5mm}

The primary benefit of Method 2 is that it can use the raw IRT data, rather
than simple ordinal comparisons, which may provide a better fit. Unlike Method
1, it also ties IRTs to the specific walk that created them. I suspect this is
not a big factor, but I don't know for sure.

\vspace{5mm}

The drawbacks are that Method 2 relies on generating a computationally
expensive sample of $Z$s as we have been doing, and that it relies on a prior
animal semantic network to judge the likelihood. The good news, though, is that
our simulations show the patten of IRTs is \emph{roughly} similar for
Erdos-Renyi and Watts-Strogatz graphs with equal number of nodes and edges. So
even though we're making assumptions about the graph properties, I suspect the
procedure will be fairly robust against it. Lastly, it's simply not as elegant;
relying on SSRs is a bit of a hack, though it saves us from making assumptions
about how hidden nodes map on to IRTs.

\section{Estimating parameters of the graph}

All of the above describes how to measure the probability of a random graph,
but not how to optimize it (i.e., determine the optimal graph/link matrix).
Using an analytical form of Method 1, we might be able to use the maximum
likelihood solution used by Jun (submitted). To be honest, I didn't quite
follow the procedure in Jun's section 2.2 for computing the optimal parameters,
so I'm not entirely sure. The other solution, which we have partially
implemented, is to use random mutations on an initial graph and follow some
sort of simulated annealing heuristic to find the best fitting graph.

\vspace{5mm}

\emph{Side note:} We can generate an initial graph using a single parameter
$\theta$, as before, though it may help to introduce an additional decay
parameter $\lambda$ which increases $\theta$ inverse-exponentially as a
function of $i$ in $X$. This may provide a better fit to the IRTs, and we could
even compute the most efficient starting parameters from training on the animal
semantic network.

\end{document}
